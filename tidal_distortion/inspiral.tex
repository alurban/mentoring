\documentclass[11pt]{article}
\usepackage{charter}
\usepackage{graphicx}
\usepackage{hyperref}
\usepackage{mdframed}
\usepackage[margin=1in]{geometry}
\usepackage{amsmath,amssymb}

\hypersetup{
	colorlinks=true,
	linkcolor=blue,
	filecolor=magenta,
	urlcolor=cyan,
}

\begin{document}

%===================================================
% Title and Author Info
%===================================================
\begin{center}
{\Large\textsc{Relativistic Orbits}} \\
\vspace{10pt}
{\large \textbf{Mentor:} Alex Urban} \\
{\small LIGO Laboratory, California Institute of Technology \\
Pasadena, CA 91125, USA \\
\href{mailto:aurban@ligo.caltech.edu}{\texttt{aurban@ligo.caltech.edu}}}
\end{center}

%%%%%%%%%%%%%%%%%%%%%%%%%%%%%%%%%%%%%%%%%%%%%%%%%%%

\section*{Orbital Death Spirals and Hydrostatic Equilibrium}
\hspace{15pt} Suppose that a compact binary with masses $m_1$, $m_2$ is in a stable circular orbit. This time, we'll model its energy loss due to gravitational waves. Figure \ref{fig:binary_diagram} illustrates this system.

\vspace{20pt}

\begin{figure}[!h]
\begin{mdframed}
\centering
\includegraphics{relativistic_orbit/binary_diagram.pdf}
\caption{\label{fig:binary_diagram}Diagram of the neutron star binary in the center of mass frame, showing its orbital separation vector ($\mathbf{r}$) and the radius ($R$) and masses ($m_1$, $m_2$) of the individual neutron stars. The azimuthal angle $\varphi$ is also indicated.}
\end{mdframed}
\end{figure}

\vspace{10pt}

As we have seen, the orbiting binary is a simple example of a system with nonzero \textit{quadrupole moment} that changes over time. According to general relativity, this means the system will emit gravitational waves, which carry orbital energy away at a rate
\begin{equation}\label{eq:GW_luminosity}
-L_{\rm GW} = \frac{dE}{dt} = - \frac{32}{5} \frac{G^4}{c^5} \frac{\mu^2 M^3}{r^5}
\end{equation}
where $M = m_1 + m_2$ is the total mass and $\mu = m_1 m_2/(m_1 + m_2)$ is the reduced mass. (This is the observed ``luminosity'' in gravitational waves, which are emitted at twice the orbital frequency.)

\vspace{1000pt}

\begin{enumerate}

\item Using the fact that the gravitational binding energy of this system at a given time is
\[ E = - \frac{GM\mu}{2r}, \]
show that the orbital separation $r$ varies like
\begin{equation}\label{eq:drdt}
\frac{dr}{dt} = - \frac{64}{5} \frac{G^3}{c^5} \frac{\mu M^2}{r^3}.
\end{equation}
Can you solve this equation for $r(t)$ the old-fashioned way (\textit{i.e.} with pen and paper)?

\item How long does it take the orbit to evolve from a gravitational wave frequency of 20 Hz up to the ISCO frequency? Express this in terms of $\mu$ and $M$, and try computing the time explicitly for $m_1 = m_2 =$ 1.4 $M_{\odot}$.

\item Numerically integrate equations \ref{eq:GW_luminosity} and \ref{eq:drdt} for a neutron star system with $m_1 = m_2 =$ 1.4 $M_{\odot}$. Plot the orbital separation, orbital velocity ($v/c$), kinetic, potential, and total energies, GW frequency, and the GW luminosity, all as a function of time. (You can start the integration from a separation of 300 km, and follow it all the way up to ISCO.) How many orbits do you wind up simulating?

\item Now, let's switch gears a bit and start building a model of very dense stellar material. To begin with, use Newtonian physics to try and show that an outward pressure on a spherical star will balance the inward pull of the star's gravity exactly when
\begin{equation}\label{eq:hydro}
\frac{dP}{dr} = - \frac{G}{r^2}\, \rho(r)\, m(r),
\end{equation}
where $\rho(r)$ is called the \textit{density profile} and
\begin{equation}\label{eq:mass}
\frac{dm}{dr} = 4 \pi r^2 \rho(r)
\end{equation}
is the mass contained in a thin spherical shell of radius $r$. (This condition, and the physical state it describes, is called \textit{hydrostatic equilibrium}.) What do $P(r)$ and $m(r)$ look like when the star is uniformly dense (so $\rho$ is a constant)?

\item \textbf{Project Point:} In astrophysics, a \textit{polytrope} loosely refers to any object whose internal pressure and density are related by
\begin{equation}\label{eq:polytrope}
P = K \rho^{\gamma}
\end{equation}
where $K$ and $\gamma$ are constants. In particular, we can imagine, say, a gas of electrons that produce an outward (nonthermal) pressure because no two of them can ever occupy the same quantum state (the Pauli exclusion principle). In this case, $\gamma =$ 5/3 and
\[ K = \frac{\hslash^2}{15\pi^2 m_e} \left( \frac{3\pi^2}{n m_N} \right)^{5/3} \]
where $m_e$ is the electron mass, $m_N$ the neutron mass, and $n$ the number of nucleons per electron. Try solving Eqs. \ref{eq:hydro} and \ref{eq:mass} numerically using this equation of state with $n=2$, which is a good model for a white dwarf star of total mass 1 $M_{\odot}$ and radius 10$^4$ km that's made mostly of neutral carbon. (We will eventually do a relativistic version of this to model the internal pressure of neutron stars.)

\end{enumerate}

%%%%%%%%%%%%%%%%%%%%%%%%%%%%%%%%%%%%%%%%%%%%%%%%%%%

\section*{Things That Make You Go, ``Hmmm....''}

\begin{enumerate}

\item How does the rate of orbital decay (Eq. \ref{eq:drdt}) compare with the orbital velocity during your model of an inspiral?

\item Can you use this to justify the approximation we talked about last time, in which we can ignore radial acceleration when we account for forces during the inspiral?

\item What physics is the Newtonian condition for hydrostatic equilibrium (Eq. \ref{eq:hydro}) failing to capture?

\end{enumerate}

%%%%%%%%%%%%%%%%%%%%%%%%%%%%%%%%%%%%%%%%%%%%%%%%%%%
%%%%%%%%%%%%%%%%%%%%%%%%%%%%%%%%%%%%%%%%%%%%%%%%%%%
\vspace{1000pt}
%%%%%%%%%%%%%%%%%%%%%%%%%%%%%%%%%%%%%%%%%%%%%%%%%%%
%%%%%%%%%%%%%%%%%%%%%%%%%%%%%%%%%%%%%%%%%%%%%%%%%%%

\section*{Solution}

\begin{enumerate}

\item From the gravitational binding energy of the binary, we can take the time derivative and see that
\[ \frac{dE}{dt} = -\frac{d}{dt}\left(\frac{GM\mu}{2r}\right) = \frac{GM\mu}{2r^2} \frac{dr}{dt}. \]
From Eq. \ref{eq:GW_luminosity}, it's then clear that
\[ \frac{dr}{dt} = \frac{2r^2}{GM\mu} \frac{dE}{dt} = - \frac{64}{5} \frac{G^3}{c^5} \frac{\mu M^2}{r^3} \]
as we set out to show.

\hspace{15pt} Now, here's the cool part: isolating terms that vary with $r$ and terms that vary with $t$ on both sides, we wind up with the integrals
\[ 4 \int r^3 \, dr = - \frac{256}{5} \frac{G^3}{c^5} \mu M^2 \int dt. \]
But since there are two different variables on each side of the equation, these variables must depend on one another:
\[ r^4 - r_0^4 = - \frac{256}{5} \frac{G^3}{c^5} \mu M^2 \left(t - t_0\right) \]
where both $t_0$ and $r_0$ arise from constants of integration. Since they're not independent, we can lump them together into one constant called $t_c$ and rearrange:
\begin{equation}\label{eq:roft}
r(t) = \left( \frac{256}{5}\,\frac{G^3}{c^5}\,\mu M^2 \right)^{1/4} \left(t_c - t\right)^{1/4}.
\end{equation}
In this case, we interpret $t_c$ as the time at which the binary reaches zero separation and the objects ``coalesce,'' \textit{i.e.} $r(t_c) = 0$. In LIGO jargon, this is the \textit{coalescence time}.

\item There is another way to interpret the integral in the previous problem. Namely, the time $\Delta t$ taken for the binary to evolve from any initial separation $r$ to any (smaller) separation $r_0 < r$ is
\[ \Delta t = \frac{5}{256} \frac{c^5}{G^3} \frac{1}{\mu M^2} \left(r^4 - r_0^4\right). \]
If $r_0 = 6GM/c^2$ is the ISCO radius and $r = [GM/(\pi f_{\rm GW})^2]^{1/3}$ is the separation at some GW frequency $f_{\rm GW} < f_{\rm ISCO}$, then
\[ \Delta t = \frac{5}{256} \frac{c^5}{G^3} \frac{1}{\mu M^2} \left[\left(\frac{GM}{(\pi f_{\rm GW})^2}\right)^{4/3} - \left(\frac{6GM}{c^2}\right)^4\right]. \]
If an equal-mass neutron star system with $m_1=m_2=$ 1.4 $M_{\odot}$ starts its evolution from $f_{\rm GW}=$ 20 Hz (or a separation of about 455 km), it will reach ISCO in $\Delta t \simeq$ 158 seconds (or about 2.5 minutes).

\hspace{15pt} This timescale is important because it's the amount of time a signal from compact binary merger remains in LIGO's sensitive frequency band, which currently starts around 20-30 Hz. The binary black hole signal GW 150914 was observed to last for around 0.2 seconds between 35 and 250 Hz with $m_1 \approx m_2 \approx$ 30 $M_{\odot}$, which is roughly in the same ballpark as our prediction that it should evolve from 35 Hz to $f_{\rm ISCO}$ in $\Delta t \simeq$ 0.175 seconds. But it's not exact. What do you think we're missing? And, just for kicks, what was the luminosity of this signal?

\item I've coded up a solution to this problem in GitHub, under the \href{https://github.com/alurban/mentoring/blob/master/tidal_distortion/scripts/decaying_orbit.py}{\texttt{tidal\_deformation/scripts}} directory. Its output is plotted in Figures \ref{fig:orbit_parameters}, \ref{fig:inspiral_energy}, and \ref{fig:inspiral_diagram}. Notice that in this simulation, there are a couple of subtleties related to time resolution that can be particularly pernicious if you don't watch out for them -- especially in the last few orbits of the inspiral. Once the neutron stars start to pick up speed, you have to be careful that your step size $\Delta t$ is small enough that errors don't build up and what you get out of the simulation is sensible, smooth motion. To the same effect, you also need to be sure your step size is small enough that you get appreciably close to $r_{\rm ISCO}$ without crossing it. We're going to discuss these issues in group \emph{a lot}.

\hspace{15pt} But I also did something egomaniacal and a tiny bit evil. Every single one of the calculations you do in this simulation can actually be solved analytically -- that is, you can get an \emph{exact} formula for orbital separation, orbital phase, energy, luminosity, orbital velocity, even the gravitational wave frequnecy, all as a function of time. Let's look at one of those examples, the GW frequency. We know that for stable circular orbits, $f_{\rm GW} = 2 f_{\rm orbit} = \omega/\pi$, so
\begin{align*}
f_{\rm GW}(t) &= \frac{\omega}{\pi} = \frac{1}{\pi} \left( \frac{GM}{r^3} \right)^{1/2} \\
  &= \frac{1}{\pi} \left[ GM \left( \frac{5}{256} \, \frac{c^5}{G^3} \, \frac{1}{\mu M^2} \right)^{3/4} \left(t_c - t\right)^{-3/4} \right]^{1/2} \\
  &= \frac{5^{3/8}}{8\pi} \left( \frac{G\mathcal{M}}{c^3} \right)^{5/8} \left(t_c - t\right)^{-3/8}
\end{align*}
where $\mathcal{M} = (m_1 m_2)^{3/5}/(m_1 + m_2)^{1/5}$ is called the \textit{chirp mass}. (The GW frequency as a function of time for our simulated binary is plotted in figure \ref{fig:orbit_parameters}.) Note the distinctive shape of this time-frequency track: as time goes on, the frequency sweeps up, dramatically so toward the end of the inspiral -- almost, but not quite, entirely unlike the sound of a bird chirping! Using general relativity, we could even write down the specific form of this chirping waveform:
\begin{equation}
h(t) = \frac{2G\mu}{c^2D} \, \frac{2GM}{c^2r(t)} \, \cos\left(2\varphi(t)\right).
\end{equation}
This quantity measures the relative change in length $\Delta L/L$ of a ruler on Earth due to merging compact objects that live a distance $D$ away, where $\varphi(t) = \int\omega\,dt$ is the orbital phase. (Note that the binaries we see are typically in galaxies hundreds of millions of light-years away. Also, it is \emph{entirely} appropriate to think of LIGO as a very expensive ruler.) The number of wave cycles we observe between some time $t_0$ and the coalescence time $t_c$ is $N_{\rm cyc} = \int_{t_0}^{t_c} f_{\rm GW}(t)\,dt$, and the number of orbits completed in the same time is $N_{\rm cyc}/2$.

\hspace{15pt} So why did we bother doing a simulation at all? A couple of reasons. First, this is only the roughest approximation to the inspiral, expanded in terms of $v/c$ where $v$ is the orbital velocity. (Such an expansion is called the ``\href{https://en.wikipedia.org/wiki/Post-Newtonian_expansion}{post-Newtonian}'' framework.) When LIGO searches for real compact binary inspirals, we have to include higher-order terms in this expansion to make valid and precise measurements. Second, we can only follow the orbit up to ISCO in this scheme -- following beyond that will require full GR. Finally, and this bears repeating, making simulations has us dealing with a lot of subtleties related to time resolution and frame rates. It's enormously helpful to ground ourselves first in a situation we can easily check against pen-and-paper results.

%%%%%%%%%%%%%%%%%%%%%%%%%%%%%%
\begin{figure}[!h]
\centering
\includegraphics[scale=1]{inspiral/inspiral_orbit_parameters.pdf}
\caption{\label{fig:orbit_parameters} Properties of a simulated double neutron star orbit with $m_1=m_2=$ 1.4 $M_{\odot}$. Top: orbital separation as a function of time. Middle: Both the gravitational wave frequency (left axis) and orbital velocity (right axis); note the way the binary accelerates to over 0.4$c$ by the time it reaches ISCO. Bottom: a visualization of this binary's full gravitational waveform, shown as the relative length change $h(t) = \Delta L/L$ we would measure from a distance of 3.26 million light-years. (We call this quantity the \textit{strain} due to gravitational waves.)}
\end{figure}

\begin{figure}[!h]
\centering
\includegraphics[scale=1]{inspiral/inspiral_energies.pdf}
\caption{\label{fig:inspiral_energy} The energy scales involved during inspiral and merger of our binary neutron star system. Top: the kinetic, potential, and gravitational binding energy of the binary over time. Bottom: the apparent luminosity (or power) emitted in gravitational waves. Note that all energy scales are measured in units of $M_{\odot}c^2$, the rest mass-energy of the Sun, which is somewhere around $\sim$10$^{47}$ Joules. During GW150914, our first detected binary black hole merger, some 3 $M_{\odot}c^2$ of energy was released over a period of $\sim$0.2 seconds, making this by far the most luminous single event ever witnessed by humans. (It's about twice as bright as \emph{every other star in the observable universe at that moment combined}.)}
\end{figure}

\begin{figure}[!h]
\centering
\includegraphics[scale=1]{inspiral/inspiral_diagram.pdf}
\caption{\label{fig:inspiral_diagram} Trace of the binary neutron star inspiral, drawn in a reference frame centered on one of the neutron stars. I've cut the plot off at a max separation of 80 km so that you can clearly see the spiral shape that the orbits trace, and I've ended the simulation once the binaries reach a separation of $r_{\rm ISCO} = 6GM/c^2$. Between 40 and 2000 Hz, the system evolved from 300 to 25 km and traced about 890 complete orbits.}
\end{figure}
%%%%%%%%%%%%%%%%%%%%%%%%%%%%%%
\clearpage

\item Let's picture a spherical shell of mass $\delta m$ and radius $r$ concentric with our hypothetical star, where the total mass contained inside is $m(r)$. Then the weight felt by this shell is
\[ \delta F = - \frac{G\,m(r)}{r^2}\,\delta m = -4\pi G \, \rho(r) \, m(r) \, \delta r \]
since $\delta m = 4\pi r^2 \rho(r) \, \delta r$ according to Eq. \ref{eq:mass}. In order to be in hydrostatic equilibrium, this inward gravitational pull must be balanced by an outward force
\[ \delta F = 4\pi r^2\,\delta P \]
where $\delta P = P(r + \delta r) - P(r)$ is the pressure per unit area pushing outward against this shell. Setting the two equations equal and taking the limit as $\delta r$ goes to 0, we see that
\begin{equation}
\frac{dP}{dr} = \lim_{\delta r \rightarrow 0} \frac{P(r + \delta r) - P(r)}{\delta r} = - \frac{G}{r^2} \, \rho(r) \, m(r)
\end{equation}
which is just Eq. \ref{eq:hydro}, as we wanted to show.

\hspace{15pt} If the density profile $\rho(r) = \rho_* =$ const., then of course the mass
\[ m(r \leq R_*) = \frac{4}{3}\pi r^3 \rho_* \]
scales with the volume of the star -- that is, until we reach the radius $R_*$ of the star, at which point the density falls sharply to 0 and the mass plateus at $m_* = m(R_*)$. The radial pressure is easy to find:
\[ P(r\leq R_*) = -\frac{4}{3}\pi G\rho^2_* \int r\,dr = \frac{2}{3}\pi G\rho^2_*(R^2_* - r^2) \]
and $P(r>R_*) = 0$ outside of the star.

\hspace{15pt} Note that this model is pretty poor for a couple of reasons. First, physical materials don't remain uniformly dense inside of large bodies: the core of the Earth, for example, is crushed by the titanic weight of all the planet's layers sitting on top of it. Second, physical materials are also affected by thermodynamics, so they'll have an \textit{equation of state} relating $\rho$ and $P$, which we will of course explore in the Project Point.

\end{enumerate}

\end{document}
